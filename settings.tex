% #####################################################################
% ############################# PAQUETES ##############################
% #####################################################################

% ************************CODIFICACIÓN Y FUENTE************************
\usepackage[utf8]{inputenc}
\usepackage{mathptmx} % Defines Adobe Times Roman (or equivalent) as default text font
\pdfminorversion=7 % Por un warning por el logo en la portada

% ************************INTERNACIONALIZACIÓN************************
\usepackage[spanish,es-lcroman,es-nodecimaldot,es-tabla]{babel} % spanish, es-lcroman son para cambiar la numeración romana a mínuscula. (Por defecto, solo spanish esta en mayuscula), es-nodecimaldot es para usar "." como separador decimal

% ************************GEOMETRÍA Y MÁRGENES************************
\PassOptionsToPackage{showframe}{geometry} % Habilitar la opción de mostrar margenes en el paquete "geometry"
\usepackage[a4paper,left=2.54cm, right=2.54cm, top=2.54cm, bottom=2.54cm]{geometry} % Define los márgenes de la página

% *************************GRÁFICOS Y FIGURAS*************************
\usepackage{graphicx} % Permite incluir gráficos, \resizebox
\graphicspath{ {./img/} } % Ruta por defector para las imágenes

% ********************ENCABEZADOS Y PIES DE PÁGINA********************
\usepackage{fancyhdr} % Permite personalizar encabezados y pies de página
\setlength{\headheight}{14.5pt} % Para evitar un warning "Make it at least 14.49998pt"

% ******************************CAPTIONS******************************
%\usepackage[font=small]{caption} % Cambia el tamaño de las captions (figure y xltabular) a 10pt (ya que el documento esta definido en 12pt)
\usepackage{caption}
\captionsetup{
	format=plain,
	labelformat=simple,
	font=small,
	labelsep=colon,
	skip=6pt,
	justification=centering
}

% ****************************INTERLINEADO****************************
\usepackage{setspace} % Permite cambiar el interlineado, permite usar \doublespacing y tambien permite usar \begin{spacing} para que solo cierto texto tenga otro valor de interlineado

% *******************************TABLAS*******************************
\usepackage{xltabular} % Este paquete carga el paquete ltablex, pero mantiene el entorno tabularx actual tal como está. El nuevo entorno xltabular es una combinación de longtable y tabularx: definiciones de encabezado/pie de página, especificador de columna X y posibles saltos de página.
\usepackage{booktabs} % Para líneas horizontales profesionales: uso \toprule, \midrule, \bottomrule
\usepackage{array} % Para definir nuevos tipos de columna con alineación: uso \newcolumntype
\usepackage{tabularray,tblr-extras}
\UseTblrLibrary{caption}

% *******************************ÍNDICES*******************************
\usepackage[figurewithin=none]{caption} % Remueve el espacio entre items de distinto chapter en la Lista de figuras
\usepackage[tablewithin=none]{caption} % Remueve el espacio entre items de distinto chapter en la Lista de tablas
\usepackage{tocbibind} % Permite incluir el índice de tablas, figuras y la bibliografía al índice general
\usepackage{titletoc}% Personalización detallada de los contenidos, incluye \titlecontents

% ****************************SITUACIONALES****************************
\usepackage{lipsum} % Genera texto de relleno

% Now load showboxes:
%\usepackage{showboxes}
% Enable frame drawing around boxes:
%\ShowFrameBoxes

% ********************************OTROS********************************
\usepackage{titlesec} % Permite cambiar el tamaño, color, fuente de los títulos de chapter, section y subsection
\usepackage[hidelinks,hypertexnames=false]{hyperref} % Gestiona enlaces hipertexto, incluye (\url) y \phantomsection y % Para evitar un warning "destination with the same identifier..."
\usepackage[none]{hyphenat} % Para evitar que LaTeX corte las palabras al final de línea
\usepackage{xurl} % Permite cortar URLs largas automáticamente en saltos de línea sin errores de formato
%\usepackage{float} % Permite fijar la posición exacta de figuras/tablas con [H]
\usepackage{environ} % Permite crear el nuevo entorno my_xltabular

\usepackage{xcolor}
\definecolor{Cerulean}{rgb}{0,0.682,0.937}
\definecolor{Turbo}{rgb}{1,0.945,0.003}
\definecolor{GreenYellow}{rgb}{0.678,1,0.184}


\usepackage{colortbl}
\usepackage{multicol}
\usepackage{bigstrut}
\usepackage{float}




% #####################################################################
% ################### DEFINICIÓN DE NUEVOS COMANDOS ###################
% #####################################################################

% ------------------- COMANDO PARA IMPRIMIR EL MES ------------------
% Macro \mes: muestra el mes actual en español.
\newcommand{\mes}{%
	\ifcase\month\or Enero\or Febrero\or Marzo\or Abril\or Mayo\or Junio\or Julio\or Agosto\or Septiembre\or Octubre\or Noviembre\or Diciembre\fi}

% ------------------- CONFIGURAR FOOTERS Y HEADERS -------------------
% Se definen dos estilos, uno para el frontmatter con numeración en el centro-abajo; y mainmatterstyle, con numeración derecha-arriba
\fancypagestyle{frontmatterstyle}{ % Estilo para el frontmatter
	\fancyhf{} % Limpia el header and footer
	\fancyfoot[C]{\thepage} % En el centro del footer coloca la numeración
	\renewcommand{\headrulewidth}{0pt} % La línea se setea a grosor 0pt
}
\fancypagestyle{mainmatterstyle}{ % Estilo para el mainmatter
	\fancyhf{} % Limpia el header and footer
	\fancyhead[R]{\thepage} % En la esquina superior derecha del header coloca la numeración
	\renewcommand{\headrulewidth}{0pt} % La línea se setea a grosor 0pt
}

% ------------------ CONFIGURACIÓN DEL INTERLINEADO -------------------
\AtBeginEnvironment{figure}{\onehalfspacing} % Configura automáticamente interlineado simple en figuras
\AtBeginEnvironment{xltabular}{\onehalfspacing} % Configura automáticamente interlineado 1.5 en tablas xltabular
\AtBeginEnvironment{longtblr}{\onehalfspacing} % Configura automáticamente interlineado 1.5 en tablas xltabular

% -------- CONFIGURACIÓN DE FUENTES Y TAMAÑOS DE LOS TÍTULOS --------
% Personalización de \chapter
% - [display] : hace que el número y el título del capítulo se muestren en líneas separadas
% - Estilo del título de la sección
% - CAPÍTULO #N : Prefijo del título + Número del capítulo
% - 12pt : Espacio entre el prefijo y el título del capítulo.
% - {}: Código adicional para antes del título (vacío en este caso).
\titleformat{\chapter}[display]
{\singlespacing\normalsize\bfseries\centering}
{\MakeUppercase{\chaptertitlename} \thechapter}
{12pt}
{}

% Personalización de \section
% - Estilo del título de la sección: singlespacing, normalfont, normalsize, bfseries.
% - Prefijo del título, ej. "1.1": utiliza \thesection seguido de un punto.
% - 12pt: Espacio entre el prefijo y el título de la sección.
% - {}: Código adicional para antes del título (vacío en este caso).
\titleformat{\section}
{\singlespacing\normalsize\bfseries}
{\thesection.}
{6pt}
{}

% Personalización de \subsection
% - Estilo del título de la subsección: singlespacing, normalfont, normalsize, bfseries.
% - Prefijo del título, ej. "1.1.1": utiliza \thesubsection seguido de un punto.
% - 12pt: Espacio entre el prefijo y el título de la subsección.
% - {}: Código adicional para antes del título, si es necesario (vacío).
\titleformat{\subsection}
{\singlespacing\normalfont\normalsize\bfseries}
{\thesubsection.}
{12pt}
{}

% Reducción del espacio en blanco encima del \chapter
% - 0pt: Sin sangría a la izquierda
% - -24.5pt: Reduce el espacio vertical predeterminado antes del título.
% - 12pt: Espacio de 12pt después del título.
\titlespacing*{\chapter}{0pt}{-29.5pt}{12pt} 

% Reducción del espacio en blanco encima del \section
% - 0pt: Sin sangría a la izquierda
% - 0pt: Sin espacio vertical adicional antes del título.
% - 6pt: Espacio de 6pt después del título.
\titlespacing*{\section}{0pt}{0pt}{6pt}

% Reducción del espacio en blanco encima del \subsection
% - 0pt: Sin sangría a la izquierda.
% - 0pt: Sin espacio vertical adicional antes del título.
% - 6pt: Espacio de 6pt después del título.
\titlespacing*{\subsection}{0pt}{0pt}{6pt}

% ------------- CONFIGURACIÓN DE LA TABLA DE CONTENIDOS -------------
% Configuración de la entrada de capítulos en la tabla de contenido
% - Nivel de sección: chapter
% - 0pt: Sin sangría.
% - \vspace{6pt}\bfseries: Añade espacio vertical antes y aplica negrita al texto.
% - \MakeUppercase{\chaptername} \thecontentslabel: : Muestra el nombre del capítulo en mayúsculas y su número seguido de dos puntos.
% - {}: No se define un formato especial para entradas sin número.
% - \hfill\contentspage: Alinea el número de página a la derecha con puntos de relleno.
% - []: No hay código adicional después de la entrada.
\titlecontents{chapter}
[0pt]
{\vspace{6pt}\bfseries}
{\MakeUppercase{\chaptername} \thecontentslabel: }
{}
{\hfill\contentspage}
[]

% -------------- CONFIGURACIÓN DE UN NUEVO ENTORNO FIGURE --------------
% Crea una leyenda que incluye fuente: #1 es el título (usado en la lista y en la leyenda) y #2 la fuente.
\newcommand*{\captionsource}[2]{%
	\if\relax\detokenize{#2}\relax
	\caption[{#1}]{\centering #1.}%
	\else
	\caption[{#1}]{\centering #1.\\Fuente: Adaptado de #2}%
	\fi
}
% Define un comando para figuras con fuente y ancho parametrizado.
% #1: posición (e.g., h, t),
% #2: archivo de imagen,
% #3: título,
% #4: fuente (o cita)
% #5: ancho.
\newcommand{\sourcefigure}[5]{%
	\begin{figure}[#1]
		\centering
		\captionsetup{width=#5,belowskip=-10pt}
		\includegraphics[width=#5]{#2}
		\captionsource{#3}{#4}%
		\label{fig:#2}
	\end{figure}
}

% ----------------------- CONFIGURACIÓN DE longtblr ------------------------
\NewColumnType{Y}{Q[c,co=1]}
\DeclareTblrTemplate{contfoot-text}{normal}{Continua en la siguiente página...}
\SetTblrTemplate{contfoot-text}{normal}
\DeclareTblrTemplate{conthead-text}{normal}{\\(Continuación)}
\SetTblrTemplate{conthead-text}{normal}
\SetTblrInner[longtblr]{rowsep=0pt}

% Definición de la macro:
\newcommand{\bestTabla}[5][]{%
	\begin{longtblr}[%
		label   = {#1},%
		caption = {%
			#2%
			\if\relax\detokenize{#3}\relax
			.%  si #3 está vacío, sólo el punto
			\else
			. Fuente: Adaptado de #3% si #3 no está vacío
			\fi
		},%
		entry   = {#2}% siempre el primer argumento
		]{#4}%
		#5%
	\end{longtblr}%
}

% ------------------ EVITA EL WARNING DE UNDERFULL HBOXES ------------------
\hbadness=99999
\setlength\emergencystretch{10em}